\section*{Введение}
\addcontentsline{toc}{section}{Введение}

\indent Большое количество производственных задач можно сформулировать в виде математической задачи на экстремум. К некоторым из этих задач применимы классические методы оптимизации и математического анализа. Но для значительной части задач производства и управления подобные методы не применимы или малоэффективны. Нахождение экстремума классическими методами зачастую приводит к тому, что на следующем этапе решения приходится решать новые задачи, ещё более сложные, чем поставленные первоначально. Даже в тех случаях, когда задача формулируется, как задача математического программирования, применение необходимых условий для построения оптимального решения очень часто не приводит к нужному результату. В производственных задачах оптимизации к максимуму целевой функции легче бывает подойти поэтапно, чем аналитически решить систему уравнений, определяемую возможными ограничениями поставленной задачи. И даже, в том случае, когда задача математического программирования может быть решена, проверка найденного решения может оказаться довольно сложной и тем сложнее, чем больше аргументов у функции.

Курсовая работа посвящена моделированию оптимизационных задач средствами динамического программирования.

Целью курсовой работы является обоснование метода динамического программирования при моделировании оптимизационных задач, решение задач аналитически и с помощью программных средств.

Объект исследования -- оптимизационные модели, решаемые методом динамического программирования.

Предмет исследования -- алгоритм метода динамического программирования.

\newpage

Задачи работы:
\begin{enumerate}[wide]
    \item изучить общий подход динамического программирования;
    \item выявить оптимизационные модели, решаемые методом динамического программирования;
    \item продемонстрировать применение метода динамического программирования при решение оптимизационных задач аналитически;
    \item выполнить программную реализацию некоторых моделей.
\end{enumerate}

Актуальность, выполненной работы можно обосновать тем, что решение ряда оптимизационных задач, например, производственного управления, можно упростить, если процесс управления осуществляется поэтапно, заменив нахождение точек экстремума целевой функции многих переменных многократным нахождением точек экстремума функции одного или небольшого числа переменных, что возможно, если воспользоваться методом динамического программирования.

Результатом работы является программы решения нескольких оптимизационных задач, таких как задача о рюкзаке, задачи о замене оборудования, задача о распределении инвестиций.
