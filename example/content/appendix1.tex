\section*{Приложение А}
\label{sec:appendix1}
\addcontentsline{toc}{section}{Приложение А}

\begin{center}
    \textbf{Код программы, которая реализует решение задачи о замене оборудования.}
\end{center}

\begin{lstlisting}[language=Python]
#! Name:    Solves the problem of replacement of equipment
#! Author:  Egor Bronnikov


# Import
from typing import List, Tuple
import unittest


def replacing_equipment(n: int, year: int, s: int, P: int, r: List[int], u: List[int]) -> Tuple[List[str], int]:
    """
        @Synopsis
        def replacing_equipment(n: int, year: int, s: int, P: int, r: List[int], u: List[int]) -> Tuple[List[str], int]: ...

        @Description
        Solves the equipment replacement problem by the dynamic programming method.

        @param n: Number of years of planning
        @type n: int
        @param year: Equipment age
        @type year: int
        @param s: Residual value of equipment
        @type s: int
        @param P: New equipment costs
        @type P: int
        @param r: The value of the products produced during the year
        @type r: List[int]
        @param u: Annual costs associated with the operation of equipment
        @type u: List[int]

        @return: Equipment replacement plan for a given period of time `n` and the maximum profit (the target function value).
        @rtype: Tuple[List[str], int]
    """

    # Prepare data
    matrix = [[0 for _ in range(n+1)] for _ in range(n)]    # Matrix for Bellman function
    replacement = [0 for _ in range(n)]                     # Replacement rate

    matrix[n-1] = [max(r[t] - u[t], s - P + r[0] - u[0]) for t in range(n+1)]
    replacement[n-1] = s - P + r[0] - u[0]

    # Conditional optimization
    # Filling in the Bellman function matrix
    for i in range(n-1, 0, -1):
        for t in range(n):
            matrix[i-1][t] = max(r[t] - u[t] + matrix[i][t+1], s - P + r[0] - u[0] + matrix[i][1])
        replacement[i-1] = s - P + r[0] - u[0] + matrix[i][1]
        matrix[i-1][n] = matrix[i-1][n-1]

    # Unconditional optimization
    column = [row[year] for row in matrix]
    value = column[0]
    
    pos = year
    plan = []
    
    for i in range(n):
        if pos == matrix[i].index(replacement[i]):  # Save at this time and replace on the future
            plan.append(f"{i+1}")
            plan.append("R")
            pos = 0
        elif pos > matrix[i].index(replacement[i]): # Replace 
            plan.append("R")
            plan.append(f"{i+1}")
            pos = 0
        else:                                       # Save
            plan.append(f"{i+1}")
            pos += 1

    return plan, value


class ReplacingEquipmentTests(unittest.TestCase):

    def test1(self):
        n = 10
        y = 3
        s = 4
        P = 18
        r = [31, 30, 28, 28, 27, 26, 26, 25, 24, 24, 23]
        u = [8, 9, 9, 10, 10, 10, 11, 12, 14, 17, 18]
        self.assertEqual((["1", "2", "R", "3", "4", "5", "6", "R", "7", "8", "9", "10"], 169),
                        replacing_equipment(n, y, s, P, r, u))

    def test2(self):
        n = 8
        y = 3
        s = 3
        P = 13
        r = [16, 15, 15, 15, 13, 11, 10, 8, 7]
        u = [4, 6, 6, 6, 7, 8, 8, 10, 12]
        self.assertEqual((["1", "R", "2", "3", "4", "R", "5", "6", "7", "8"], 58),
                        replacing_equipment(n, y, s, P, r, u))


if __name__ == "__main__":
    unittest.main()
\end{lstlisting}
